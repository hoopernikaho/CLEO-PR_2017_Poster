%!TEX root =  pulse_fitting_poster.tex
\vspace{-1cm}
\begin{center}
  \begin{center} {\bf \Large \textsf {Motivation}}\end{center}
\end{center}
Standard techniques for time-tagging a detection event, based on threshold crossing or constant fraction discrimination, do not work when two pulses have a large overlap, limiting the applicability of the TES for high photon-flux detection.
%
Previously, differentiation was used to locate the steep rising edge of individual X-ray detection events when the signals overlap~\cite{Fowler:2015ef}.
%
The number of rising edges was used
to identify the number of detected photons.
%
For near-infrared (NIR) photons the signal-to-noise ratio is lower, and timing accuracy is affected by the bandwidth limitations necessary to reject false positives.
%
In this work, we estimate the arrival time using a two-level discriminator that is inherently robust against noise.
We show how it
improves photon number discrimination for NIR continuous wave (CW) sources,
and propose a scheme to determine the arrival time and pulse amplitude of the underlying pulses.